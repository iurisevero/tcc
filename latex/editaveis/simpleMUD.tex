\chapter[Evolução e Manutenção SimpleMUD]{Evolução e Manutenção SimpleMUD}

Este capítulo tem como objetivo relatar o processo de evolução e manutenção do código SimpleMUD, 
disponibilizado no livro MUD Game Programming.

\section{Primeiras alterações}

As primeiras alterações realizadas no código foram feitas para corrigir erros de compilação. 
A documentação do SimpleMUD define três passos para compilação do programa:

\begin{enumerate}
    \item \mint{bash}|make libs|
    \item \mint{bash}|make simplemud|
    \item \mint{bash}|make link|
\end{enumerate}

Sendo o primeiro para compilar as bibliotecas externas fornecidas pelo autor, 
o segundo para compilar o MUD e o terceiro para linkar (substituir por ligar?) 
os arquivos objeto gerados. Ao tentar executá-los diversos erros foram levantados.

\subsection{make libs}

O SimpleMUD utiliza três bibliotecas em sua codificação. 
Uma para funções básicas, BasicLib (forma certa de citar?), uma para lidar com \textit{sockets}, 
SocketLib, e uma para \textit{threads}, ThreadLib. Ao executar o comando de compilação 
foram encontrados erros na biblioteca básica e na de \textit{sockets}.

\subsubsection{BasicLib}

Os erros levantados na biblioteca básica se concentravam em torno das funções matemáticas 
de logarítmo, cosseno e seno, que são utilizadas no arquivo BasicLibRandom.h 
(como citar esse arquivo direito?)(Bloco de código \ref{lst:mathError}).

\begin{listing}[!ht]
    \begin{minted}[
        frame=lines,
        bgcolor=LightGray,
        fontsize=\footnotesize,
        linenos,
        breaklines
    ]{bash}
    ../Libraries/BasicLib/BasicLibRandom.h: In member function 'double BasicLib::normal_generator<inclusive, generator>::operator()()':
    ../Libraries/BasicLib/BasicLibRandom.h:132:31: error: there are no arguments to 'log' that depend on a template parameter, so a declaration of 'log' must be available [-fpermissive]
      132 |         m_rho = sqrt(-2 * log(1- m_rho2));
          |                           ^~~
    ../Libraries/BasicLib/BasicLibRandom.h:132:31: note: (if you use '-fpermissive', G++ will accept your code, but allowing the use of an undeclared name is deprecated)
    ../Libraries/BasicLib/BasicLibRandom.h:140:36: error: there are no arguments to 'cos' that depend on a template parameter, so a declaration of 'cos' must be available [-fpermissive]
      140 | turn m_rho * ( m_valid ? cos(2 * pi * m_rho1) : sin( 2 * pi * m_rho1)) * m_sigma + m_mean;
          |                          ^~~
    ../Libraries/BasicLib/BasicLibRandom.h:140:59: error: there are no arguments to 'sin' that depend on a template parameter, so a declaration of 'sin' must be available [-fpermissive]
      140 | ? cos(2 * pi * m_rho1) : sin( 2 * pi * m_rho1)) * m_sigma + m_mean;
          |                          ^~~
    \end{minted}
\caption{Exemplo de erros da biblioteca básica}
\label{lst:mathError}
\end{listing}
        

Para corrigí-los foi necessário incluir a biblioteca cmath (forma certa d citar?), 
que faz parte da Biblioteca GNU ISO C++.

\subsubsection{SocketLib}

A biblioteca de \textit{sockets} apresentou dois erros recorrentes. O primeiro foi com a função 'memset',
no arquivo SocketLibSocket.cpp, cujo a correção foi indicada pelo compilador (Bloco de código \ref{lst:memsetError}). 
Após a inclusão da biblioteca cstring os erros foram resolvidos.

\begin{listing}[!ht]
    \begin{minted}[
        frame=lines,
        bgcolor=LightGray,
        fontsize=\footnotesize,
        linenos,
        breaklines
    ]{bash}
    ../Libraries/SocketLib/SocketLibSocket.cpp: In member function 'void SocketLib::DataSocket::Connect(SocketLib::ipaddress, SocketLib::port)':
    ../Libraries/SocketLib/SocketLibSocket.cpp:141:9: error: 'memset' was not declared in this scope
      141 |         memset( &(m_remoteinfo.sin_zero), 0, 8 );
          |         ^~~~~~
    ../Libraries/SocketLib/SocketLibSocket.cpp:9:1: note: 'memset' is defined in header '<cstring>'; did you forget to '#include <cstring>'?
        8 | #include "SocketLibSocket.h"
      +++ |+#include <cstring>
        9 | 
    \end{minted}
\caption{Exemplo de erro levantado pela função 'memset'}
\label{lst:memsetError}
\end{listing}

O segundo erro foi resultado da declaração do tipo 'clistitr', apresentado no bloco de código \ref{lst:clistitrError}. 
Como o próprio compilador aponta, a correção do erro é possível a partir da adição da 
palavra chave \textit{typename} antes da definição de 'clistitr' e após a palavra chave \textit{typedef}. 
Essa correção foi necessária devido à dependência no escopo do tipo definido\footnote{
    Uma explicação resumida desse erro pode ser encontrada no StackOverflow, \href{https://stackoverflow.com/questions/3311633/nested-templates-with-dependent-scope/3311640\#3311640}{Nested templates with dependent scope}
}.

\begin{listing}[!ht]
    \begin{minted}[
        frame=lines,
        bgcolor=LightGray,
        fontsize=\footnotesize,
        linenos,
        breaklines
    ]{bash}
    In file included from ../Libraries/SocketLib/Connection.h:17,
                        from ../Libraries/SocketLib/Telnet.h:14,
                        from ../Libraries/SocketLib/Telnet.cpp:9:
    ../Libraries/SocketLib/ConnectionManager.h: At global scope:
    ../Libraries/SocketLib/ConnectionManager.h:37:13: error: need 'typename' before 'std::__cxx11::list<SocketLib::Connection<protocol> >::iterator' because 'std::__cxx11::list<SocketLib::Connection<protocol> >' is a dependent scope
       37 |     typedef std::list< Connection<protocol> >::iterator clistitr;
          |             ^~~
          |             typename 
    ../Libraries/SocketLib/ConnectionManager.h:110:17: error: 'clistitr' has not been declared
      110 |     void Close( clistitr p_itr );
          |                 ^~~~~~~~
    \end{minted}
\caption{Exemplo de erro levantado pela declaração de 'clistitr'}
\label{lst:clistitrError}
\end{listing}

O bloco de código \ref{lst:clistitrCorrection} apresenta a revisão feita.

\begin{listing}[!ht]
    \begin{minted}[
        frame=lines,
        bgcolor=LightGray,
        fontsize=\footnotesize,
        linenos,
        breaklines
    ]{cpp}
    // ConnectionManager.h, linha 37
    // Código com erro
    typedef std::list< Connection<protocol> >::iterator clistitr;

    // Código revisado
    typedef typename std::list< Connection<protocol> >::iterator clistitr;
    \end{minted}
\caption{Revisão da declaração de 'clistitr'}
\label{lst:clistitrCorrection}
\end{listing}

\subsection{make simplemud}

Assim como a compilação das bibliotecas, o comando para compilar o MUD também apresentou 
erros ao ser executado. Alguns semelhantes aos já resolvidos, como o exemplo apresentado 
no bloco de código \ref{lst:knownError}, o que tornou a correção mais rápida.

\begin{listing}[!ht]
    \begin{minted}[
        frame=lines,
        bgcolor=LightGray,
        fontsize=\footnotesize,
        linenos,
        breaklines
    ]{bash}
    SimpleMUD/EntityDatabase.h:47:25: error: need 'typename' before 'SimpleMUD::EntityDatabase<datatype>::container::iterator' because 'SimpleMUD::EntityDatabase<datatype>::container' is a dependent scope
      47 |         iterator( const container::iterator& p_itr ) // copy constructor
         |                         ^~~~~~~~~
         |                         typename 
    \end{minted}
\caption{Exemplo de erro conhecido}
\label{lst:knownError}
\end{listing}

Todos erros levantados na primeira tentativa de utilização do comando se referiam ao arquivo
EntityDatabase.h e, em sua maioria, aconteciam ao tentar declarar o iterator do tipo 'container',
bloco de código \ref{lst:itrError}, definido pelo próprio autor.

\begin{listing}[!ht]
    \begin{minted}[
        frame=lines,
        bgcolor=LightGray,
        fontsize=\footnotesize,
        linenos,
        breaklines
    ]{bash}
    SimpleMUD/EntityDatabase.h: In constructor 'SimpleMUD::EntityDatabase<datatype>::iterator::iterator(const int&)':
    SimpleMUD/EntityDatabase.h:49:34: error: 'itr' was not declared in this scope
      49 |             container::iterator& itr = *this; // also needed because VC6 sucks
         |                                  ^~~
    \end{minted}
\caption{Exemplo de erro ao declarar iterador do tipo 'container'}
\label{lst:itrError}
\end{listing}

Dentre as diversas possibilidades de correção, a opção adotada foi a de definir um novo tipo para
esse iterador, o tipo 'containeritr', seguindo o padrão adotado nas bibliotecas utilizadas.
Essa escolha foi feita devido a quantidade de locais onde o iterador era utilizado no código analisado.
Após a definição do novo tipo, todos acessos à 'container::iterator' e  
'std::map<entityid,datatype>::iterator' foram atualizados. O resultado das alterações pode ser visto nos
blocos de código \ref{lst:containeritrOri} e \ref{lst:containeritrFix}.

\begin{listing}[!ht]
    \begin{minted}[
        frame=lines,
        bgcolor=LightGray,
        fontsize=\footnotesize,
        linenos,
        breaklines
    ]{cpp}
    // EntityDatabase.h
    template< class datatype >
    class EntityDatabase
    {
    public:

        typedef std::map<entityid, datatype> container;

        // --------------------------------------------------------------------
        //  The inner iterator class, used to iterate through the database.
        // --------------------------------------------------------------------
        class iterator : public container::iterator
        {
        public:

            // --------------------------------------------------------------------
            // NOTE: the constructors are needed as a result of VC6 sucking.
            // Have I mentioned that VC6 sucks yet?
            // --------------------------------------------------------------------
            iterator() {};  // default constructor
            iterator( const container::iterator& p_itr ) // copy constructor
            {
                container::iterator& itr = *this; // also needed because VC6 sucks
                itr = p_itr;
            }
    \end{minted}
\caption{Acesso ao iterador do tipo 'container' antes da correção}
\label{lst:containeritrOri}
\end{listing}

\begin{listing}[!ht]
    \begin{minted}[
        frame=lines,
        bgcolor=LightGray,
        fontsize=\footnotesize,
        linenos,
        breaklines
    ]{cpp}
    // EntityDatabase.h
    template< class datatype >
    class EntityDatabase
    {
    public:

        typedef std::map<entityid, datatype> container;
        typedef typename std::map<entityid, datatype>::iterator containeritr;

        // --------------------------------------------------------------------
        //  The inner iterator class, used to iterate through the database.
        // --------------------------------------------------------------------
        class iterator : public containeritr
        {
        public:

            // --------------------------------------------------------------------
            // NOTE: the constructors are needed as a result of VC6 sucking.
            // Have I mentioned that VC6 sucks yet?
            // --------------------------------------------------------------------
            iterator() {};  // default constructor
            iterator( const containeritr& p_itr ) // copy constructor
            {
                containeritr& itr = *this; // also needed because VC6 sucks
                itr = p_itr;
            }
    \end{minted}
\caption{Correção do acesso ao iterador do tipo 'container'}
\label{lst:containeritrFix}
\end{listing}


Após corrigidos os erros levantados pelo arquivo EntityDatabase.h, uma segunda tentatiza de 
executar o comando foi feita e resultou em novos erros (bloco de código \ref{lst:templateError}).

\begin{listing}[!ht]
    \begin{minted}[
        frame=lines,
        bgcolor=LightGray,
        fontsize=\footnotesize,
        linenos,
        breaklines
    ]{bash}
    g++ -I../Libraries *.cpp -c;
    g++ -I../Libraries ./SimpleMUD/*.cpp -c;
    ./SimpleMUD/EnemyDatabase.cpp:25:28: error: specializing member 'SimpleMUD::EntityDatabaseVector<SimpleMUD::EnemyTemplate>::m_vector' requires 'template<>' syntax
       25 | std::vector<EnemyTemplate> EntityDatabaseVector<EnemyTemplate>::m_vector;
          |                            ^~~~~~~~~~~~~~~~~~~~~~~~~~~~~~~~~~~
    ./SimpleMUD/EnemyDatabase.cpp:28:27: error: specializing member 'SimpleMUD::EntityDatabase<SimpleMUD::Enemy>::m_map' requires 'template<>' syntax
       28 | std::map<entityid, Enemy> EntityDatabase<Enemy>::m_map;
       |                           ^~~~~~~~~~~~~~~~~~~~~
    ./SimpleMUD/ItemDatabase.cpp:20:26: error: specializing member 'SimpleMUD::EntityDatabase<SimpleMUD::Item>::m_map' requires 'template<>' syntax
       20 | std::map<entityid, Item> EntityDatabase<Item>::m_map;
          |                          ^~~~~~~~~~~~~~~~~~~~
    ./SimpleMUD/PlayerDatabase.cpp:24:28: error: specializing member 'SimpleMUD::EntityDatabase<SimpleMUD::Player>::m_map' requires 'template<>' syntax
       24 | std::map<entityid, Player> EntityDatabase<Player>::m_map;
          |                            ^~~~~~~~~~~~~~~~~~~~~~
    ./SimpleMUD/RoomDatabase.cpp:25:19: error: specializing member 'SimpleMUD::EntityDatabaseVector<SimpleMUD::Room>::m_vector' requires 'template<>' syntax
       25 | std::vector<Room> EntityDatabaseVector<Room>::m_vector;
          |                   ^~~~~~~~~~~~~~~~~~~~~~~~~~
    ./SimpleMUD/StoreDatabase.cpp:20:27: error: specializing member 'SimpleMUD::EntityDatabase<SimpleMUD::Store>::m_map' requires 'template<>' syntax
       20 | std::map<entityid, Store> EntityDatabase<Store>::m_map;
          |                           ^~~~~~~~~~~~~~~~~~~~~
    make: *** [makefile:24: simplemud] Error 1
    \end{minted}
\caption{Erros da sintaxe de template}
\label{lst:templateError}
\end{listing}

Os novos erros foram decorrentes da forma como as varíaveis estáticas das entidades do bancos de dados
foram declaradas nas classes que possuiam herança com \textit{EntityDatabase} ou com \textit{EntityDatabaseVector}. 
Um mesmo erro para vários arquivos, como é possível observar no bloco de código \ref{lst:templateError}. 

A solução adotada foi a adição da palavra chave 'template<>', utilizando como argumento a mesma 
classe passada para o template da classe base\footnote{Solução encontrada na documentação da IBM 
\href{https://www.ibm.com/docs/en/zos/2.4.0?topic=only-static-data-members-templates-c}
{Static data members and templates (C++ only)}}, antes da declaração da variável. 
O bloco de código \ref{lst:templateFix} apresenta o resultado da correção.

\begin{listing}[!ht]
    \begin{minted}[
        frame=lines,
        bgcolor=LightGray,
        fontsize=\footnotesize,
        linenos,
        breaklines
    ]{cpp}
    // EnemyDatabase.cpp, linha 27
    // Código original
    // declare the static map of the enemy instance database.
    std::map<entityid, Enemy> EntityDatabase<Enemy>::m_map;

    // Código corrigido
    // declare the static map of the enemy instance database.
    template< class Enemy >
    std::map<entityid, Enemy> EntityDatabase<Enemy>::m_map;
    \end{minted}
\caption{Correção da declaração das variáveis estáticas}
\label{lst:templateFix}
\end{listing}

O comando make simplemud funcionou corretamente após as alterações, assim como o comando make link.
Não foram encontrados novos erros.

Após conseguir completar o processo de compilação do código e execução do MUD, 
foi iniciada uma análise em busca de trechos do código que poderiam ser atualizados
para seguirem os padrões estabelecidos no \textit{C++ Core Guidelines}.


\section{Alterações arquiteturais do Banco de Dados}

Para o banco de dados, Peton utiliza uma abordagem com dicionários, vetores, classes e funções estáticos, 
auxiliados por arquivos para persistir os dados. Dessa forma ele conseguiu uma estrutura que o 
permitiu acessar os dados de forma global e sem a necessidade de instânciar objetos das classes 
criadas \cite{penton2003mud}.

\begin{citacao}
    Because the classes are static, you should be able to access
    a single database within the game globally, without
    worrying about instantiating it.\cite{penton2003mud}
\end{citacao}

No entanto, essa abordagem é ingênua de implementação do padrão \textit{Singleton} 
e desencadeia alguns problemas, como a ordem de inicialização de objetos estáticos 
globais em unidades de compilação diferentes ser indefinida, o que pode resultar
em um objeto global referindo-se a outro quando este ainda não foi inicializado \cite{nesteruk2018design}.

A implementação clássica do padrão \textit{Singleton}, sugerida por Nesteruk, é uma forma
de melhorar a implementação feita por Peton, tornando-a mais segura e sem perder sua intenção inicial.
Essa foi a opção adotada para melhorar o código analisado.

Foram feitas alterações significativas nas classes EntityDatabase, EntityDatabaseVector, 
EnemyTemplateDatabase, EnemyDatabase, ItemDatabase, PlayerDatabase, RoomDatabase e StoreDatabase. 
Os códigos \ref{lst:enemyDatabaseHeaderOri}-\ref{lst:enemyDatabaseOri} apresentam um exemplo 
de como foram feitas as implementações de Penton, enquanto os códigos 
\ref{lst:enemyDatabaseHeaderUpdate}-\ref{lst:enemyDatabaseUpdate} apresentam as modificações realizadas.

\begin{listing}[!ht]
    \begin{minted}[
        frame=lines,
        bgcolor=LightGray,
        fontsize=\footnotesize,
        linenos,
        breaklines
    ]{cpp}
    // EnemyDatabase.h
    class EnemyTemplateDatabase : public EntityDatabaseVector<EnemyTemplate>
    {
    public:
        static void Load();
    };  // end class EnemyTemplateDatabase


    class EnemyDatabase : public EntityDatabase<Enemy>
    {
    public:
        static void Create( entityid p_template, room p_room );
        static void Delete( enemy p_enemy );
        static void Load();
        static void Save();

    };  // end class EnemyDatabase
    \end{minted}
\caption{Cabeçalho das classes EnemyTemplateDatabase e EnemyDatabase originais}
\label{lst:enemyDatabaseHeaderOri}
\end{listing}

\begin{listing}[!ht]
    \begin{minted}[
        frame=lines,
        bgcolor=LightGray,
        fontsize=\footnotesize,
        linenos,
        breaklines
    ]{cpp}
    // EnemyDatabase.cpp
    // declare the static vector of the enemy template database.
    std::vector<EnemyTemplate> EntityDatabaseVector<EnemyTemplate>::m_vector;

    // declare the static map of the enemy instance database.
    std::map<entityid, Enemy> EntityDatabase<Enemy>::m_map;
    \end{minted}
\caption{Classes EnemyTemplateDatabase e EnemyDatabase originais}
\label{lst:enemyDatabaseOri}
\end{listing}

A variável m\_map é inicialmente declarada na classe EntityDatabase, que serve como classe base para
o banco de dados, no entanto, por ser estática, é necessário redefini-la em toda classe derivada para 
evitar problemas de linkagem ao compilar os arquivos. O mesmo vale para variável m\_vector, da classe EntityDatabaseVector.

Outra peculiaridade dessa implementação original é a necessidade de todas funções serem estáticas, uma vez que 
não há instancia das classes. Ambas situações foram extinguidas com a nova implementação.

\begin{listing}[!ht]
    \begin{minted}[
        frame=lines,
        bgcolor=LightGray,
        fontsize=\footnotesize,
        linenos,
        breaklines
    ]{cpp}
    // EnemyDatabase.h
    class EnemyTemplateDatabase : public EntityDatabaseVector<EnemyTemplate>
    {
    public:
        static EnemyTemplateDatabase& GetInstance();
        EnemyTemplateDatabase(EnemyTemplateDatabase const&) = delete;
        EnemyTemplateDatabase(EnemyTemplateDatabase&&) = delete;
        EnemyTemplateDatabase& operator=(EnemyTemplateDatabase const&) = delete;
        EnemyTemplateDatabase& operator=(EnemyTemplateDatabase &&) = delete;

        void Load();

    private:
        EnemyTemplateDatabase(){}
    };  // end class EnemyTemplateDatabase


    class EnemyDatabase : public EntityDatabase<Enemy>
    {
    public:
        static EnemyDatabase& GetInstance();
        EnemyDatabase(EnemyDatabase const&) = delete;
        EnemyDatabase(EnemyDatabase&&) = delete;
        EnemyDatabase& operator=(EnemyDatabase const&) = delete;
        EnemyDatabase& operator=(EnemyDatabase &&) = delete;

        void Create( entityid p_template, room p_room );
        void Delete( enemy p_enemy );
        void Load();
        void Save();

    private:
        EnemyDatabase(){}
    };  // end class EnemyDatabase
    \end{minted}
\caption{Cabeçalho das classes EnemyTemplateDatabase e EnemyDatabase atualizadas}
\label{lst:enemyDatabaseHeaderUpdate}
\end{listing}

\begin{listing}[!ht]
    \begin{minted}[
        frame=lines,
        bgcolor=LightGray,
        fontsize=\footnotesize,
        linenos,
        breaklines
    ]{cpp}
    // EnemyDatabase.cpp
    EnemyTemplateDatabase& EnemyTemplateDatabase::GetInstance()
    {
        static EnemyTemplateDatabase enemyTemplateDatabase;
        return enemyTemplateDatabase;
    }

    EnemyDatabase& EnemyDatabase::GetInstance()
    {
        static EnemyDatabase enemyDatabase;
        return enemyDatabase;
    }
    \end{minted}
\caption{Classes EnemyTemplateDatabase e EnemyDatabase atualizadas}
\label{lst:enemyDatabaseUpdate}
\end{listing}

Com a instanciação das classes, as funções e atributos deixaram de ser estáticos e os acessos 
as entidades do banco de dados passou ser feito a partir da chamada GetInstance (Código \ref{lst:enemyDelete}). 
Diversas alterações pequenas foram feitas em outros arquivos para adaptá-los ao novo padrão adotado.

\begin{listing}[!ht]
    \begin{minted}[
        frame=lines,
        bgcolor=LightGray,
        fontsize=\footnotesize,
        linenos,
        breaklines
    ]{cpp}
    // Game.cpp, linha 1296
    // Código original
    EnemyDatabase::Delete( p_enemy );

    // Código pós alterações
    EnemyDatabase::GetInstance().Delete( p_enemy );
    \end{minted}
\caption{Exemplo de deleção de um inimigo pré e pós da atualização das entidades}
\label{lst:enemyDelete}
\end{listing}

\section{Outras alterações}

Durante a refatoração, foram encontrados trechos de código que poderiam ser atualizados, tanto
para a utilização de novas funcionalidades adicionadas no C++ quanto para se obter um código mais limpo.

\subsection{EntityDatabase iterator}

A classe EntityDatabase tem como objetivo servir de template para as entidades do banco de dados. 
Nela ocorre o agrupamento de um dicionário e funções de busca, assim como a abstração dessa 
estrutura em um novo tipo, chamado \textit{container} e seu iterador, \textit{containeritr}. 
Por fim, uma classe \textit{iterator} é criada para sobrescrever o iterador padrão da estrutura 
definida na STL. O código \ref{lst:entityDatabase} apresenta partes dessa classe.

\begin{listing}[!ht]
    \begin{minted}[
        frame=lines,
        bgcolor=LightGray,
        fontsize=\footnotesize,
        linenos,
        breaklines
    ]{cpp}
    // EntityDatabase.h
    template< class datatype >
    class EntityDatabase
    {
    public:

        typedef std::map<entityid, datatype> container;
        typedef typename std::map<entityid, datatype>::iterator containeritr;
        // --------------------------------------------------------------------
        //  The inner iterator class, used to iterate through the database.
        // --------------------------------------------------------------------
        class iterator : public containeritr{}

    protected:
        std::map<entityid, datatype> m_map;
    };  // end class EntityDatabase
    \end{minted}
\caption{Trechos da classe EntityDatabase}
\label{lst:entityDatabase}
\end{listing}

Na construção do iterador interno ocorre a sobrescrita de dois operadores, o operador 
\textit{*} e o \textit{->}, utilizados para desreferência e seleção de elemento por ponteiro, 
respectivamente. No entanto, as novas funcionalidades atribuidas para esses operadores utilizam 
o operador \textit{->} da classe base, porém o acesso a esse operador é perdido, uma vez que ele 
é sobrescrito. Para contornar essa situação, Peton cria um cópia do ponteiro da classe derivada e a 
"converte" para classe base, seguido pela chamada do operador desejado, o que resultou em um 
código não muito claro e de difícil interpretação, código \ref{lst:innerIteratorEntityDatabase}.

\begin{listing}[!ht]
    \begin{minted}[
        frame=lines,
        bgcolor=LightGray,
        fontsize=\footnotesize,
        linenos,
        breaklines
    ]{cpp}
    // EntityDatabase.h
    class iterator : public containeritr
    {
    public:

        // --------------------------------------------------------------------
        // NOTE: the constructors are needed as a result of VC6 sucking.
        // Have I mentioned that VC6 sucks yet?
        // --------------------------------------------------------------------
        iterator() {};  // default constructor
        iterator( const containeritr& p_itr ) // copy constructor
        {
            containeritr& itr = *this; // also needed because VC6 sucks
            itr = p_itr;
        }

        // --------------------------------------------------------------------
        //  "dereferences" the iterator to return a reference to the
        //  object that it points to.
        // --------------------------------------------------------------------
        inline datatype& operator*()
        {
            containeritr& itr = *this; // also needed because VC6 sucks
            return itr->second;
        }

        // --------------------------------------------------------------------
        //  the "pointer-to-member" operator, which will allow you to use it on
        //  iterators like this: itr->Function();
        // --------------------------------------------------------------------
        inline datatype* operator->()
        {
            containeritr& itr = *this; // also needed because VC6 sucks
            return &(itr->second);
        }
    };  // end iterator inner class
    \end{minted}
\caption{Iterador interno da classe EntityDatabase}
\label{lst:innerIteratorEntityDatabase}
\end{listing}

Todas as funções apresentadas foram atualizadas. Os construtores passaram a utilizar um 
método moderno, enquanto os operadores passaram a utilizar a função \textit{static\_cast}, 
que atinge o mesmo resultado de acessar os operadores da classe base, porém de uma forma mais limpa. 
O código \ref{lst:innerIteratorEntityDatabaseRefc} apresenta a classe \textit{iterator} 
refatorada.

\begin{listing}[!ht]
    \begin{minted}[
        frame=lines,
        bgcolor=LightGray,
        fontsize=\footnotesize,
        linenos,
        breaklines
    ]{cpp}
    // EntityDatabase.h
    class iterator : public containeritr
    {
    public:
        iterator() {};  // default constructor
        iterator( const containeritr& p_itr ) : containeritr( p_itr ){} // copy constructor

        // --------------------------------------------------------------------
        //  "dereferences" the iterator to return a reference to the
        //  object that it points to.
        // --------------------------------------------------------------------
        inline datatype& operator*() 
        {
            return static_cast<containeritr>(*this)->second;
        }

        // --------------------------------------------------------------------
        //  the "pointer-to-member" operator, which will allow you to use it on
        //  iterators like this: itr->Function();
        // --------------------------------------------------------------------
        inline datatype* operator->()
        {
            return &(static_cast<containeritr>(*this)->second);
        }
    };  // end iterator inner class
    \end{minted}
\caption{Refatoração do iterador interno da classe EntityDatabase}
\label{lst:innerIteratorEntityDatabaseRefc}
\end{listing}

\subsection{Estruturas de Repetição}

A partir do C++11 foi adicionada a estrutura de repetição for baseada em um intervalo definido. 
O \textit{Range-based for loop} é usado como um equivalente mais legível ao loop for tradicional 
operando em uma faixa de valores, como todos os elementos em um contêiner \cite{rangebasedfor}. 

A iteração por todos elementos de um contêiner é facilmente encontrada em diversos segmentos do 
código do SimpleMUD, como apresentado no código \ref{lst:whileloop}, e podem ser atualizados utilizando
o \textit{range-based for loop}, código \ref{lst:rangebasedfor}.

\begin{listing}[!ht]
    \begin{minted}[
        frame=lines,
        bgcolor=LightGray,
        fontsize=\footnotesize,
        linenos,
        breaklines
    ]{cpp}
    // GameLoop.cpp, linha 151
    void GameLoop::PerformHeal()
    {
        PlayerDatabase::iterator itr = PlayerDatabase::GetInstance().begin();
        while( itr != PlayerDatabase::GetInstance().end() )
        {
            if( itr->Active() )
            {
                itr->AddHitpoints( itr->GetAttr( HPREGEN ) );
                itr->PrintStatbar( true );
            }
            ++itr;
        }
    }
    \end{minted}
\caption{Função PerformHeal da classe GameLoop}
\label{lst:whileloop}
\end{listing}

\begin{listing}[!ht]
    \begin{minted}[
        frame=lines,
        bgcolor=LightGray,
        fontsize=\footnotesize,
        linenos,
        breaklines
    ]{cpp}
    // GameLoop.cpp, linha 143
    void GameLoop::PerformHeal()
    {
        for(auto& player : PlayerDatabase::GetInstance()){
            if( player.Active() )
            {
                player.AddHitpoints( player.GetAttr( HPREGEN ) );
                player.PrintStatbar( true );
            }
        }
    }
    \end{minted}
\caption{Função PerformHeal da classe GameLoop atualizada}
\label{lst:rangebasedfor}
\end{listing}

Para atualização das estruturas de repetição foi necessário a alteração de 6 arquivos, totalizando
9 funções refatoradas e a redução de 21 linhas de código.
