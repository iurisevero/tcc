\section{Multi-user Dungeons}

% The SimpleMUD: This is a very
% simple MUD (the name doesn't lie!), but it's a good start for
% understanding how you can combine a simple data system
% with a networking reaction system. In the SimpleMUD, I use
% C++ exclusively to code the physical and logical aspects of
% the game, and simple ASCII data files to store the data level 
% of the game.

% \cite{penton2003mud}

A relação entre \textit{Multi-user Dungeons} (MUDs) e mundos virtuais surge desde antes da internet, 
na época de 1970, quando o primeiro MUD foi lançado. Os mundos virtuais costumam ser chamados de MUDs 
porque MUD foi o nome do primeiro mundo virtual a prosperar. No entanto, enquanto mundos virtuais são 
ambientes considerados autocontidos, MUDs são, principalmente, mundos virtuais baseados em texto \cite{bartle2004designing,bartle2015multi}. 
Em outras palavras, todo MUD é um mundo virtual, porém nem todo mundo virtual é um MUD.

De acordo com o artigo "Multi-User Dungeons (MUDs)" de Bartle, há oito critérios estabelecidos para 
determinar se um mundo virtual é considerado um MUD, os quais são:

\begin{enumerate}
    \item Ele opera por meio de um conjunto de regras automáticas que permitem aos jogadores efetuar 
    alterações no ambiente virtual; essas regras são conhecidas como sua física. Salas de bate-papo e 
    Dungeons \& Dragons não são MUDs porque não possuem física;
    \item Os jogadores interagem com o mundo através do canal de um único objeto que eles controlam 
    sozinhos e que está “dentro” do mundo; este objeto é seu caráter. Jogos de estratégia não são MUDs 
    porque os jogadores não agem por meio de um personagem;
    \item A interação ocorre em tempo real. Jogos baseados em turnos não são MUDs porque o tempo entre 
    os movimentos é muito perceptível.
    \item Vários jogadores podem compartilhar o mesmo ambiente virtual, no qual qualquer um deles pode 
    fazer alterações. Jogos para um jogador não são MUDs porque o mundo não é compartilhado
    \item As mudanças no mundo devem ser capazes de durar mesmo que todos jogadores encerrem suas 
    sessões; isso é conhecido como persistência. Jogos de tiro em primeira pessoa não são MUDs porque 
    não são persistentes
    \item O mundo não deve ser o mundo real. A realidade não é um MUD porque é o mundo real.
    \item Tem mecânica de jogo codificada em sua física. O Second Life não é um MUD porque não tem jogabilidade.
    \item Tem uma interface principalmente textual. World of Warcraft não é um MUD porque é gráfico.
\end{enumerate}

Atualmente, os critérios 1 a 6 são abrangidos pelo termo mundo virtual, que é imparcial em relação aos critérios 7 e 8 \cite{bartle2015multi}.

No aspecto do desenvolvimento, além de todo o trabalho de comunicação, geralmente há três partes em uma 
\textit{game engine} de MUD: a física, a lógica e a de dados. A parte física é a que controla a existencia 
e movimentação de items, personagens, salas... Basicamente tudo que pode ser representado como um objeto 
físico, os quais normalmente são definidos como entidades. A parte lógica é responsável pelo que acontece 
na camada física, como o que um personagem faz quando atacado ou quando recebe um item. Personagens precisam 
tomar decisões, items precisam realizar ações quando são equipados, entre outras coisas. Por fim, a parte 
de dados é a que define todas entidades físicas do jogo, por exemplo, sempre que você carrega um mapa na 
camada física, o mapa é carregado da camada de dados.

A forma como essas três partes são desenvolvidas nos MUDs variou bastante com tempo. Inicialmente 
elas eram armazenadas dentro do código, o que dificultava alterações no jogo, uma vez que era necessário 
ir no código, modificá-lo, recompilá-lo, reiniciar o MUD e o executar novamente.

A camada de dados foi a primeira a se tornar flexível. Salvar os dados em arquivos possibilitava 
alterar as informações das entidades sem precisar reiniciar o MUD. Após isso veio a flexibilização 
da camada lógica, que, essencialmente, passou a permitir a alteração de como todo o jogo funciona 
enquanto ele está rodando. A possibilidade de uma camada física flexível também existe, porém não é 
algo que se popularizou, visto que a necessidade de uma camada física flexível não existe, uma vez 
que a inserção de novos tipos de entidade é algo raro de acontecer \cite{penton2003mud}.

\subsection{Evolução dos MUDs}

Bartle, em seu livro "Design Virtual Words", dividi a evolução dos MUDs em cinco eras, 
que vão desde 1978 até 2004, quando o livro foi publicado.

Durante a Primeira Era, que aconteceu entre 1978 e 1985, foi feito o desenvolvimento do primeiro jogo 
virtual chamado MUD (Multi-User Dungeon) criado por Roy Trubshaw, com o auxílio de Richard Bartle, 
na Universidade de Essex na Inglaterra no final de 1978. O jogo foi escrito originalmente em MACRO-10 
assembler e depois dividido em duas partes: a \textit{game engine}, escrita em BCPL, e o mundo do jogo escrito 
em uma linguagem criada por Trubshaw chamada MUDDL. A ideia era criar um jogo com múltiplos mundos 
que funcionaria no mesmo motor. O sucesso do jogo foi tão grande que seu nome acabou virando a definição 
de todo o gênero e, posteriormente, ele passou a ser chamado de MUD1.

A Segunda Era dos MUDs, entre 1985 e 1989, foi marcada pelo experimento constante tanto na criação do 
mundo do jogo quanto na criação da \textit{game engine}, com muitas contribuições originais vindo do 
grupo MirrorWorld no sistema IOWA (Input/Output World of Adventure). Foi decidido reescrever MUD1 do 
zero como MUD2 e uma nova linguagem, MUDDLE (Multi-User Dungeon Definition LanguagE), foi desenvolvida 
especificamente para escrever MUDs. A maioria dos MUDs da segunda era foram programados por entusiastas 
em casa, pois poucas instituições acadêmicas no Reino Unido forneciam recursos de computação como 
Universidade de Essex. A exceção foi AberMUD, escrito pela Universidade de Wales em Aberystwyth por 
Alan Cox em 1987, que foi posteriormente portado para C, o que gerou oportunidades de grandes 
avanços no gênero, pois permitiu a execução em ambientes \textit{Unix}. Praticamente todas as 
questões-chave do design de MUDs foram identificadas na primeira e na segunda era.

A Terceira Era, entre 1989 e 1995, foi marcada por um período de grande crescimento no número de 
pessoas experimentando mundos virtuais. AberMUD se espalhou rapidamente entre departamentos de ciência 
da computação universitários, gerando cópias idênticas em milhares de máquinas Unix. Isso resultou em 
vários imitadores, sendo os principais TinyMUD, LPMUD e DikuMUD. De acordo com uma pesquisa sobre o 
tráfego na rede NSFnet em 1993, cerca de 10\% dos bits pertenciam a MUDs, ou seja, antes da chegada da 
World Wide Web, os MUDs representavam cerca de 10\% da Internet.

A Quarta Era (1995 - 1997) e a Quinta Era (1997 - Presente) foram marcadas por avanços na forma como a 
internet era comercializada e, consequentemente, a forma como os MUDs eram comercializados. Além disso 
também houveram avanços na tecnologia relacionada a interfaces e gráficos, o que permitiu o surgimento 
dos primeiros mundos virtuais 2D e, posteriormente, 3D. No final de 1979, o primeiro mundo virtual 
gráfico totalmente funcional foi lançado, Avatar. No início dos anos 1990, a Kesmai Corporation fez um jogo de simulador de vôo multiplayer, Air Warrior, 
que tinha clientes para PC, Atari ST, Commodore Amiga e Apple Macintosh. O Meridian 59 foi projetado por 
Mike Sellers e Damion Schubert, e pretendia se tornar o primeiro “3D MUD”, objetivo o qual conseguiu \cite{bartle2004designing}.