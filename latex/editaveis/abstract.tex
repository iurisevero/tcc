\begin{resumo}[Abstract]
  \begin{otherlanguage*}{english}
    Multi-User Dungeons (MUDs) are a genre of games that originated in the 1980s with the emergence of 
    the game MUD1, which gave name to the genre. These games consist of text-based virtual worlds that 
    can be explored by multiple players in an online environment. Although they were widely explored 
    until the early 2000s, with technological advances, they were replaced by graphical games. However, 
    the codes used to develop them are still valuable to students in technology areas, who can use them 
    to explore multidisciplinary subjects. This final project aims to analyze the SimpleMUD code base, 
    developed by Ron Penton, and evolve it according to current technology standards, specifically C++. 
    This will make it possible to make it available for other students to explore. The code evolution 
    will be adaptive, aiming to update it without altering its functionality, following the C++ core
    guidelines established by Bjarne Stroustrup and Herb Sutter, as well as other patterns found in the 
    community. As a result, a more secure code will be obtained, in a development environment that will 
    facilitate future maintenance and evolution by other students
    \vspace{\onelineskip}
 
    \noindent 
    \textbf{Key-words}: MUD. Software evolution. Software maintenance. C++.
  \end{otherlanguage*}
\end{resumo}
