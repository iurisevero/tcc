\begin{resumo}
    \textit{Multi-User Dungeons} (MUDs) são um gênero de jogos que se originaram na década de 1980 com 
    o surgimento do jogo MUD1, que deu nome ao gênero. Esses jogos consistem em mundos virtuais baseados 
    em texto que podem ser explorados por múltiplos jogadores em um ambiente \textit{online}. Embora 
    tenham sido amplamente explorados até o início da década de 2000, com o avanço tecnológico, foram 
    substituídos por jogos gráficos. No entanto, os códigos utilizados para desenvolvê-los ainda são 
    valiosos para estudantes das áreas de tecnologia, que podem utilizá-los para explorar assuntos 
    multidisciplinares. Este trabalho de conclusão de curso tem como objetivo analisar o código base do 
    SimpleMUD, desenvolvido por Ron Penton, e evoluí-lo de acordo com os padrões atuais da tecnologia 
    utilizada em seu desenvolvimento, o C++. Dessa forma, será possível disponibilizá-lo para que outros 
    estudantes possam explorá-lo. A evolução do código será adaptativa, visando atualizá-lo sem alterar 
    suas funcionalidades, seguindo as diretrizes principais de C++ estabelecidas por Bjarne Stroustrup e 
    Herb Sutter, bem como outros padrões encontrados na comunidade. Como resultado, será obtido um código 
    mais seguro, em um ambiente de desenvolvimento que facilitará futuras manutenções e evoluções por 
    parte de outros estudantes.

    \vspace{\onelineskip}
    
    \noindent
    \textbf{Palavras-chave}: MUD. Evolução software. Manutenção de software. C++.
\end{resumo}
