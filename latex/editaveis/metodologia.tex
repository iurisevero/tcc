\chapter[Metodologia]{Metodologia}
\label{chapter:metodologia}

Para este estudo serão utilizadas as definições e tipos de manutenção e evolução de software 
propostos por Chapin, em "Types of software evolution and software maintenance". 
A partir da instruções listadas em seu artigo, é possível definir que será realizada uma evolução 
adaptativa, uma vez que o software será evoluido para uma versão mais moderna de C++, o que irá 
alterar as práticas de design e implementação, porém sem alterar suas funcionalidades.

Apesar do tipo de evolução dominante ser a adaptativa, também será possível ver nas alterações realizadas 
graus de manutenção avaliativa e para aperfeiçoamento, como explicado na seção \ref{section:chapinTypes}, 
pela figura \ref{fig:relationAmongTypes}. Para realizar a evolução proposta será necessário analisar o 
código original e, após a alterações, o código resultante será mais seguro e manutenível, abrangindo 
os dois tipos citados \cite{chapin2001types}.

\begin{citacao}
    % A atualização de sistemas antigos é difícil. No entanto, acreditamos que um programa que usa uma regra é menos sujeito a erros e mais fácil de manter do que um que não usa \cite{stroustrup2022c++}.
    Upgrading old systems is hard. However, we do believe that a program that uses a rule is less error-prone and more maintainable than one that does not \cite{stroustrup2022c++}.
\end{citacao}

A seção \ref{section:tcc1} irá apresentar a abordagem adotada para realização da primeira etapa deste TCC, 
enquanto a seção \ref{section:tcc2} será responsável por expor o planejamento realizado para segunda etapa.

\section{TCC 1}
\label{section:tcc1}

\textcolor{red}{Pensar em um título para essa seção}

O desenvolvimento da primeira fase deste projeto não seguiu uma metodologia estabelecida. Como as fontes 
de código avaliadas eram antigas, o objetivo principal definido foi fazer com que compilassem 
corretamente. Para tal, foi realizado um processo iterativo de tentativas de compilação e correção de 
erros, continuando até que o funcionamento do MUD fosse alcançado, como foi o caso com o SimpleMUD, ou 
até que não houvesse mais progresso nas correções, como foi observado com o Dyrt.

A compilação e execução do Dyrt foram realizadas em um ambiente Manjaro, versão 5.6, com o compilador 
GCC, na versão 12.2. Já o SimpleMUD foi compilado e executado em uma máquina virtual, 
utilizando a imagem Ubuntu 20.04.5 LTS (Focal Fossa), com o compilador G++, na versão 12.2. A máquina 
virtual foi gerada e configurada com o uso do software Vagrant, na versão 2.3.4. O arquivo de 
configuração utilizado pelo Vagrant está disponível no apêndice \ref{apendice:vagrantFile}.

Após o sucesso na compilação e execução do SimpleMUD, foi iniciado um novo ciclo de atividades para a 
sua evolução e manutenção. Este ciclo incluía a pesquisa das diretrizes básicas de C++, conforme 
apresentadas por Bjarne Stroustrup e Herb Sutter, bem como outras práticas recomendadas, como aquelas 
apresentadas pelo canal de Youtube de Jason Turner, intitulado C++ Weekly\footnote{\url{https://www.youtube.com/@cppweekly}}. 
Além disso, foi realizada uma análise do código para identificar trechos que pudessem ser melhorados. 
Depois de realizada a análise, as atualizações eram implementadas e documentadas. A análise e o 
planejamento para atualização do código eram realizados em pareamentos com o professor Edson.

Durante a primeira etapa, a falta de uma metodologia formal de desenvolvimento resultou em uma evolução 
mal documentada do código. Não houve uma estratégia clara de melhoria, as modificações foram realizadas 
baseadas nas percepções e conhecimentos prévios dos desenvolvedores envolvidos, e não foram coletados 
dados quantitativos para comprovar a efetividade dessas alterações.

\section{TCC 2}
\label{section:tcc2}

\textcolor{red}{Pensar em um título para essa seção}

A segunda etapa deste TCC tem como objetivo principal organizar o ambiente de desenvolvimento do 
SimpleMUD, de modo a garantir uma evolução e manutenção eficazes. De acordo com Turner, são cruciais para 
o desenvolvimento adequado em C++: Ambiente de construção contínuo; Utilização do máximo de compiladores 
possíveis; Uma estrutura de teste organizada; Análise, relatórios e rastreamento da cobertura de teste; 
O máximo de análise estática possível; Análise de tempo de execução durante os testes; Teste de fuzz; 
Lançamento com reforço de segurança habilitado. \cite{turnerC++Weekly350}. No entanto, suprir todas essas necessidades para um desenvolvimento adequado não é algo trivial e, para isso, 
será necessário um planejamento prévio.

Para o planejamento é possível utilizar metodologias e artefatos já conhecidos da engenharia de software, 
como o Kanban ou a criação de um \textit{backlog} com as atividades a serem realizadas. A priorização do 
que será feito também é essencial, uma vez que nem todos requisitos elicitados por Turner são necessariamente 
significativos no contexto abordado.

Por fim, uma vez que o ambiente de desenvolvimento for implementado, será possível realizar uma avaliação 
quantitativa da base de código analisada e elaborar um plano de melhoria apropriado. Além disso, as 
modificações realizadas na primeira etapa poderão ser avaliadas, uma vez que todo o código foi versionado 
com a ferramenta Git e armazenado no Github.

\section{Ferramentas Utilizadas}

Nesta seção serão detalhadas as ferramentas utilizadas para o desenvolvimento deste TCC. É importante 
destacar que outras ferramentas serão acrescentadas ao decorrer da segunda etapa, visto que os passos 
estabelecidos por Turner requerem o uso de vários recursos externos.

\begin{citacao}
    % Se você está desenvolvendo às cegas, sem nenhuma orientação de ferramenta, você está utilizando C++ errado \cite{turnerC++Weekly350}.
    If you are developing blindly, without any tool guidance, you are doing C++ wrong \cite{turnerC++Weekly350}.
\end{citacao}

\subsection{Git}

Git é uma ferramenta gratuita e de código aberto de controle de versão distribuído, desenvolvida por 
Linus Torvalds em 2005. Alguns dos objetivos desse sistema incluem velocidade, design simples, forte 
suporte ao desenvolvimento não linear (milhares de ramificações paralelas), totalmente distribuído e 
capaz de lidar eficientemente com projetos de grande porte, como o núcleo do Linux (velocidade e tamanho 
de dados). Desde o seu nascimento em 2005, o Git evoluiu e amadureceu para ser fácil de usar e ainda 
retêm essas qualidades iniciais \cite{chacon2014pro}.

\subsection{GitHub}

GitHub é uma plataforma de gerenciamento de código baseada na web que fornece soluções completas para 
armazenamento, controle de versão e colaboração em projetos. Ela hospeda repositórios do Git e oferece 
aos desenvolvedores ferramentas eficientes como linha de comando, problemas (discussões encadeadas), 
\textit{pull requests}, revisão de código e acesso a uma ampla coleção de aplicativos grátis e pagos no 
GitHub Marketplace. Oferece planos gratuitos e pagos para usuários pessoais e organizações, permitindo 
que colaboradores trabalhem juntos em repositórios públicos ou privados com acesso a diferentes conjuntos 
de recursos. Com uma comunidade de 15 milhões de desenvolvedores, juntamente com integrações eficientes, 
GitHub revoluciona a forma como o software é construído \cite{github}.

\subsection{Vagrant}

Vagrant é uma ferramenta destinada à construção e gestão de ambientes de desenvolvimento completos 
baseados em máquinas virtuais. Com uma abordagem fácil de usar e focada na automação, Vagrant contribui 
para a redução do tempo de configuração do ambiente de desenvolvimento e para o aumento da correspondência 
entre o desenvolvimento e a produção. As máquinas virtuais podem ser provisionadas por diversos provedores, 
tais como o VirtualBox, VMware, AWS, e outros. As ferramentas de provisionamento padrão da indústria, 
como shell scripts, Chef ou Puppet, podem ser empregadas para a instalação e configuração automatizadas 
do software na máquina virtual.
