\chapter*[Introdução]{Introdução}
\addcontentsline{toc}{chapter}{Introdução}

\section{Contextualização}

\textit{Multi-user Dungeon} é um gênero de jogos que surgiu em meados dos anos 80 e, desde sua origem, 
se demonstraram como sofwares complexos, diversificados e adaptáveis. Os MUDs foram os primeiros mundos 
virtuais a fazerem sucesso e serviram como base para o desenvolvimento de diversos outros softwares, 
como MMORPGs e alguns simuladores \cite{bartle2004designing,penton2003mud}.

Devido à sua natureza, os MUDs requerem o uso de tecnologias distintas para sua operação e abrangem uma 
ampla gama de disciplinas da engenharia de software. O componente \textit{multiplayer} exige habilidades 
na programação de redes, enquanto a gestão dos jogadores necessita de conhecimentos em bancos de dados ou 
manipulação de arquivos. Além disso, é indispensável ter familiaridade com \textit{multithreading}, 
estruturas de dados e outros aspectos para o desenvolvimento do jogo propriamente dito.

Com a evolução das capacidades gráficas dos computadores, os jogadores migraram dos MUDs baseados em texto 
para mundos virtuais com gráficos bidimensionais e tridimensionais, assim como os desenvolvedores \cite{bartle2004designing}, 
porém a forma como o software era estruturado se manteve. Quase todos os jogos estilo mundo virtual 
possuem estruturas similares por baixo dos panos \cite{penton2003mud}.

\section{Justificativa/Motivação}

Apesar de terem se tornado ultrapassados, o código base dos MUDs ainda pode ser considerado material valioso 
para fins educacionais. Disponibilizar um MUD em funcionamento em uma instituição acadêmica permitiria aos 
alunos a oportunidade de jogar, testar e analisar o jogo, a fim de compreender mais profundamente seu 
desenvolvimento, arquitetura e outros aspectos relevantes \textcolor{red}{(sem fonte)}.

Além disso, também seria possível evoluir o MUD, de maneira que ele se torne mais seguro, escalável e até 
mesmo mais lúdico. Em sua conclusão, Penton discute as possíveis melhorias que podem ser feitas nos MUDs 
que ele desenvolveu, e destaca em vários momentos como o código pode ser personalizado pelo desenvolvedor 
para que o jogo possua um estilo único.

Existem muitos aspectos que precisam ser considerados na administração de um MUD, e gerenciá-los pode ser 
uma excelente oportunidade de desenvolvimento para estudantes de cursos relacionados a tecnologia. 
É importante planejar as questões relacionadas à implantação do jogo, incluindo a conexão com a internet 
e a garantia da segurança dessa conexão com os jogadores, a implementação de novas funcionalidades, a 
realização de manutenção e correção de erros,  e assim por diante. No entanto, para que isso seja possível, 
é necessário que o software tenha sido atualizado e esteja funcionando com as tecnologias atuais.

\textcolor{red}{(Verificar se não tem problema fazer essa citação)}
\begin{citacao}
    % Este livro não é o princípio e o fim de todos os MUDs. Eu apenas arranhei a superfície do que você pode fazer... ...mas não há absolutamente nenhuma razão para que as tecnologias MUD não possam ser estendidas infinitamente \cite{penton2003mud}.
    This book isn't the be-all and end-all of MUDs. I've only scratched the surface of what you can do... ...but there's absolutely no reason why MUD technologies can't be extended infinitely \cite{penton2003mud}.
\end{citacao}

% Frases originais
% This book isn't the be-all and end-all of MUDs. I've only
% scratched the surface of what you can do, and you should
% consider a number of things when thinking about the future.

% Typically MUDs never see more than a few dozen people at
% a time, a hundred or so at most; but there's absolutely no
% reason why MUD technologies can't be extended infinitely.

\section{Objetivos}

Este trabalho tem como objetivo principal tornar a base de códigos de um MUD implantável, para que outros 
estudantes possam explorá-lo e evoluí-lo conforme desejarem. Para isso será necessário:

\begin{itemize}
    \item Selecionar uma base de códigos de um MUD;
    \item Analisar a base selecionada;
    \item Atualizá-la para que ela funcione utilizando tecnologias atuais;
    \item Criar um ambiente de desenvolvimento onde seja possível realizar um desenvolvimento contínuo da base;
    \item Relizar manutenções necessária.
\end{itemize}

Deve-se destacar que este trabalho está sendo desenvolvido em paralelo com uma iniciação científica cujo 
objetivo é modernizar e implementar um MUD, utilizando uma base de dados relacional. \textcolor{red}{(Coloco isso?)}

\section{Estrutura do Trabalho}

Este trabalho será estruturado da seguinte forma:

\begin{itemize}
    \item Capítulo \ref{chapter:fundamentacao} - Fundamentação teórica: apresenta os conceitos essenciais 
    para compreender o trabalho em questão, com enfoque na história e desenvolvimento dos MUDs, bem como 
    na evolução e manutenção de software;
    \item Capítulo \ref{chapter:metodologia} - Metodologia: exibe o planejamento metodologico utilizado para 
    a evolução e manutenção da base de códigos selecionada, assim como o planejamento da metodologia que 
    será adotada para a próxima etapada do desenvolvimento desse trabalho;
    \item Capítulo \ref{chapter:resultadosParciais} - Resultados parciais: ilustra os resultados parciais 
    adquiridos até o momento.
\end{itemize}